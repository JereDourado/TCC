\include{preambulo}

\addcontentsline{tocx}{chapter}{Referências Bibliográficas}
\begin{thebibliography}{20}

\bibitem{BIERMAN}
BIERMAN, H. Scott; FERNANDEZ, Luis;
\textbf{Teoria Dos Jogos},
2ª edição, 
Pearson, São Paulo - SP.
2012.

\bibitem{SARTINI}
SARTINI, Brígida Alexandre; GARBUGIO, Gilmar; BORTOLOSSI, Humberto José; SANTOS, Polyane Alves; BARRETO, Larissa Santana;
\textbf{Uma Introdução a Teoria dos Jogos},
II Bienal da SBM, Universidade Federal da Bahia, 25 a 29 de outubro de 2004.

\bibitem{NEUMANN}
NEUMANN, John von;
\textbf{Zur Theorie der Gesellschaftsspiele}. Mathematische Annalen, vol. 100. Traduzido por S. Bargmann: \textbf{On the Theory of Games of Stategy em Contributions to the Theory of Games}, vol. 4, A. W. Tucker e R. D. Luce (editores), Princeton University Press, 1959.

\bibitem{COURNOT}
COURNOT, Augustin A.; 
\textbf{Recherches sur les Principes Mathématiques de la Théorie des Richesses}, 1838. Traduzido por N. T. Bacon em \textbf{Researches into the Mathematical Principles of the Theory of Wealth}, 
McMillan, New York, 1927.

\bibitem{ZERMELO}
ZERMELO, Ernst;
\textbf{Über eine Anwendung der Mengdenlehre auf die theories des Schachspiels.} Atas do Décimo Quinto Congresso Internacional de Matemáticos, vol. 2, 1913.

\bibitem{NEUMANN1}
NEUMANN, John von e MORGENSTERN, Oscar; \textbf{Theory of Games and Economic Behavior}. 
Princeton University Press, 1944.

\bibitem{NASH3}
NASH Jr, John F.; \textbf{Two-person Cooperative Games}. 
Econometrica, 1953.

\bibitem{NASH}
NASH Jr, John F.; \textbf{Non-Cooperative Games}. PhD. Thesis. 
Princeton University Press, 1950.

\bibitem{NASH1}
NASH Jr, John F.; \textbf{Non-Cooperative Games}. 
Annals of Mathematics, 1951.

\bibitem{NASH2}
NASH Jr, John F.; \textbf{The Bargaining Problem}. 
Econometrica, 1950.

\end{thebibliography}

\end{document}