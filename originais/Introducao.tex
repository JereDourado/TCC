%\include{preambulo}


\pagenumbering{arabic}
\setcounter{page}{10}

\addcontentsline{toc}{chapter}{Introdução}
\chapter*{Introdução}

O que leva as pessoas a honrarem seus compromissos? Não ser responsável com os acordos é quase sempre a opção mais tentadora, no entanto nos sentimos na obrigação de cumprir o que combinamos, em parte pode ser pelo medo de consequências futuras e em parte porque velamos pelo nosso "nome na praça" ou nosso \textit{score}, em ambos os casos no pano de fundo tem a mesma coisa, a reputação. No primeiro caso se é só o medo que nos motiva, se tivéssemos certeza de que ninguém saberia dos nossos acordos não cumpridos, não teríamos mais medo. O mesmo se aplica a segunda opção. Então porque criamos, se não criamos, como surgiu naturalmente algo tão inconveniente como a reputação?

A \textbf{teoria dos jogos} se propõe a responder essa e muitas outras questões sobre nossas interações com o mundo a nossa volta. Nela, métodos matemáticos são usados para modelar desde situações corriqueiras do dia-a-dia até a maneira como se comportam as grandes multinacionais, desde uma rede social até um site de compra e venda.

Este trabalho usa algumas ferramentas da teoria dos jogos para explicar o surgimento, a evolução e por incrível que pareça os benefícios da reputação. A mesma coisa que aparentemente e pensando apenas unilateralmente, nos causa desconforto, vista sob outra ótica ela é uma das bases que solidificam nossas relações.

%\end{document}