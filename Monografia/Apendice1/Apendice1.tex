%\include{../preambulo/preambulo.tex}

\addcontentsline{toc}{chapter}{Apêndice 1}
\section*{Apêndice 1}

{\scriptsize
\begin{verbatim}
p=input('População Inicial P = ');
coe=input('Percentual de Indivíduos que Cooperam K = ');
n=input('Número de Gerações N = ');
pc=coe.*p; % criação do grupo que coopera
pd=p-pc; % criação do grupo que não coopera
rc=zeros(1,n); % vetor para armazenar quantos cooperava a cada geração
rd=zeros(1,n); % vetor para armazenar quantos não cooperava a cada geração
a=rand; % gerador aleatorio da estrategia do jogador a
b=rand; % gerador aleatorio da estrategia do jogador b
T=[3 0; 5 1];
for i=1:n % jogo sendo executado n vezes
    if a<=coe % escolha da jogada do jogador a
        ja=1;
    else
        ja=0;
    end
    if b<=coe % escolha da jogada do jogador b
        jb=1;
    else
        jb=0;
    end
    if ja==1 && jb==1
        pc=pc+2.*T(1,1);
        pd=pd+T(1,2);
    elseif ja==1 && jb==0
        pc=pc+T(1,2);
        pd=pd+T(2,1);
    elseif ja==0 && jb==1
        pc=pc+T(1,2);
        pd=pd+T(2,1);
    elseif ja==0 && jb==0
        pc=pc+T(1,2);
        pd=pd+2.*T(2,2);
    end
    pn=pc+pd; % soma da população nessa geração
    coe=pc./pn; % atualizando o percentual do grupo que coopera
    pc=coe.*p; % atualizando o grupo que coopera
    pd=p-pc; % atualizando o grupo que não coopera
    rc(i)=pc; % armazenando quantos cooperou nessa geração
    rd(i)=pd; % armazenando quantos cooperou nessa geração
    a=rand; % gerador aleatorio da estrategia do jogador a
    b=rand; % gerador aleatorio da estrategia do jogador a
end
aux=figure(1); % gerando o gráfico
set(aux,'Name','Evolução da Proporção')
plot(1:n,rc,'g',1:n,rd,'r')
legend('Grupo A - Indivíduos que cooperam','Grupo B - Indivíduos que não cooperam')
\end{verbatim}}



%\end{document}