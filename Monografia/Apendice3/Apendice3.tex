%\documentclass[a4paper,12pt,dvipdfm]{report}
\usepackage[brazil]{babel}
%\usepackage[latin1]{inputenc}
\usepackage[utf8]{inputenc}
\usepackage{indentfirst}
\usepackage[pdftex]{color,graphicx}
\usepackage{geometry}
\geometry{top=3cm ,bottom=2cm,left=2.5cm,right=2cm}
%\usepackage{dingbat}

%\usepackage[pdftex]{color,graphicx}
\usepackage{amstext}
\usepackage{amscd}
\usepackage{amsfonts}
\usepackage{float}
\usepackage{textcomp}
\usepackage{amssymb}
%\usepackage{subfigure}
\usepackage{amsmath}
\usepackage{amscd}
%\usepackage{graphics}
%\usepackage{picinpar}
\usepackage{multicol}
\usepackage{multirow}
%\usepackage{epigraph}
%\usepackage[utf8]{inputenc}
%\usepackage{natbib}
%\usepackage{setspace}
\usepackage{mathrsfs}
\usepackage{lscape}
%\usepackage{pdfpages}
\usepackage[normalem]{ulem}
%\usepackage{tikz}
\usepackage[all]{xy}
\usepackage{enumerate}
\usepackage{mathdesign}



%\setcounter{secnumdepth}{5}
%\setcounter{tocdepth}{5}

%\usepackage[brazil]{babel}
%\usepackage[latin1]{inputenc}
%\usepackage[T1]{fontenc}
%\usepackage{indentfirst}
%\usepackage[dvips]{color}
%\usepackage{caption}
%\usepackage{float}
%\usepackage[nottoc]{tocbibind} %inclui referencias no indice.
%\usepackage{enumerate}
%\usepackage{amsmath,amsfonts,amssymb}
%\usepackage{graphicx}
%\usepackage{verbatim}
\usepackage{amsthm}
%\usepackage{natbib}
%\usepackage{subfigure}

%\usepackage{setspace}

\setlength{\parindent}{1.5cm}

\renewcommand{\baselinestretch}{1.5}
\newcommand{\nulo}{\varnothing}
\newcommand{\x}{\times}
%\newcommand{\ital}[1]{\textit{#1}}
%\newcommand{\negr}[1]{\textbf{#1}}
\newcommand{\duascolunas}[2]{\begin{minipage}{7cm} #1 \end{minipage}\hfill\begin{minipage}{7cm} #2 \end{minipage}\\\\} 
\newcommand {\expo}[1]{\exp{\left(#1\right)}}
\newcommand {\expi}[1]{\exp{i\left(#1\right)}}
\newcommand{\arc}[1]{\ensuremath{\overset{\frown}{\raisebox{0pt}[6pt]{#1}}}}


\providecommand{\sin}{} \renewcommand{\sin}{\hspace{2pt}\textrm{sen\hspace{2pt}}}
\providecommand{\tan}{} \renewcommand{\tan}{\hspace{2pt}\textrm{tg\hspace{2pt}}}
\providecommand{\arctan}{} \renewcommand{\arctan}{\hspace{2pt}\textrm{arctg\hspace{2pt}}}
\providecommand{\arcsin}{} \renewcommand{\arcsin}{\hspace{2pt}\textrm{arcsen\hspace{2pt}}}

\theoremstyle{definition}

\newcommand*{\mes}{\ifthenelse{\the\month < 2}{Janeiro}
                  {\ifthenelse{\the\month < 3}{Fevereiro}
                  {\ifthenelse{\the\month < 4}{Março}
                  {\ifthenelse{\the\month < 5}{Abril}
                  {\ifthenelse{\the\month < 6}{Maio}
                  {\ifthenelse{\the\month < 7}{Junho}
                  {\ifthenelse{\the\month < 8}{Julho}
                  {\ifthenelse{\the\month < 9}{Agosto}
                  {\ifthenelse{\the\month < 10}{Stembro}
                  {\ifthenelse{\the\month < 11}{Outubro}
                  {\ifthenelse{\the\month < 12}{Novembro}{Dezembro}}}}}}}}}}}}








\newcommand {\sen}[1]{\sin{\left(#1\right)}}
\newcommand {\cossen}[1]{\cos{\left(#1\right)}}
\newcommand {\tg}[1]{\tan{\left(#1\right)}}
\newcommand {\cotg}[1]{\cot{\left(#1\right)}}
\newcommand {\seca}[1]{\sec{\left(#1\right)}}
\newcommand {\cossec}[1]{\csc{\left(#1\right)}}



\newcommand{\E}{\xi}
\newcommand {\refe}[1]{(\ref{#1})}
%\onehalfspace
%\newcommand {\L}{\mathscr{L}}
\newcommand {\Ima}[1]{\mathrm{Im}{\left[#1\right]}}
\newcommand {\F}{\mathscr{F}}
%\newcommand {\L}{\mathscr{L}}
\newcommand {\om}{\Omega}
\newcommand {\fii}{\varphi}
\newcommand {\lap}{\Delta}
\newcommand {\gra}{\nabla}
\newcommand {\pc}{\vskip 1pc}
\newcommand {\fim}{\nl\rightline{$\square$}\vskip 2pc}
\newcommand {\nl}{\newline}
\newcommand {\cl}{\centerline}
\newcommand {\R}{\mathbb{R}}
\newcommand {\N}{\mathbb{N}}
\newcommand {\Z}{\mathbb{Z}}
\newcommand {\V}{\mathcal{V}^{hp}}
\newcommand {\Q}{\mathbb{Q}}
%\newcommand {\F}{\mathbb{F}}
\newcommand {\G}{\mathbb{G}}
\newcommand {\C}{\mathbb{C}}
\newcommand {\Ss}{\mathbb{S}}
\newcommand {\Ph}{\mathcal{P}_{\!h}}
\newcommand {\B}{\mathcal{B}}
\newcommand {\f}{\mathcal{F}}
\newcommand {\Lh}{\mathcal{L}}
\newcommand {\La}{\Lambda}
\newcommand{\Ri}{\Rightarrow}
\newcommand{\Li}{\Leftarrow}
\newcommand{\lr}{\Longleftrightarrow}
\newcommand{\dis}{\displaystyle}
\newcommand{\lon}{\longrightarrow}
\newcommand{\nin}{/\!\!\!\!\!\in}
%\newcommand {\la}{\lambda}
\newcommand {\al}{\alpha}
\newcommand {\bt}{\beta}
\newcommand {\til}{\widetilde}
\newcommand {\lb}{\linebreak}
\newcommand {\esp}{\hskip 1pc}
\newcommand {\be}{\nl\cl }
\newcommand {\normf}[3]{\Big| \!\! \; \Big|  \dfrac{#1}{#2} \Big| \!\! \; \Big|_{#3}}
\newcommand {\norma}[2] {{\parallel  \! #1 \!  \parallel}_{#2}}
\newcommand {\normp}[1] {{|\!|\!| #1 |\!|\!|}_{\! \Ph}}
\newcommand {\adsum}{\addcontentsline{toc}{subsection}}
\newcommand {\T}{\mathcal{T}}
\newcommand {\ovl}{\overline}
\newcommand {\pref}[1]{(\ref{#1})}
\newcommand {\prcr}[2]{(#1\cup #2)_{\al,L}\rtimes\N}
\newcommand {\tp}[2]{\T(#1\cup #2)}
\newcommand{\mdc}{\text{mdc}}



\newcommand{\funcao}[5]{\begin{array}{cccc}
#1:&\!\!\!#2 & \rightarrow & #3 \\
  &\!\!\! #4 & \mapsto & #5
\end{array}}
\newcommand{\n}{{\bf n}}
\newcommand{\soma}[2]{\displaystyle\sum_{#1}^{#2}}
\newcommand {\flecha}[1] {\stackrel{#1 \rightarrow \infty}\longrightarrow}
\newcommand{\canto}[1]{\begin{flushright} #1 \end{flushright}}
\newcommand{\fd}{\vspace{-0,5cm} \begin{flushright} $\square$ \end{flushright} \vspace{-0,5cm}}
\newcommand{\der}{\partial}
%\newcommand{\sen}{{\rm  \ \! sen}}
\newcommand{\dem}{\hspace{-1.5cm}\textit{Demonstra��o: }}
\newtheorem{teorema}{Teorema}[chapter]

\newtheorem{corolario}[teorema]{Corol\'ario}
\newtheorem{lema}[teorema]{Lema}
\newtheorem{proposicao}[teorema]{Proposi\c{c}\~ao}

%\newtheorem{lema}{Lema}[chapter]
%\newtheorem{corolario}{Corol�rio}[chapter]
%\newtheorem{proposicao}{Proposi��o}[chapter]

\newtheorem{definicao}{Defini\c{c}\~{a}o}[section]
\newtheorem{propriedade}{Propriedades}[section]
\newtheorem*{obs}{Observa\c{c}\~{a}o}
\newtheorem{ex}{Exemplo}[section]
\newtheorem*{solucao}{Solu\c{c}\~{a}o}
\newtheorem*{demo}{Demonstra\c{c}\~{a}o}
%\everymath{\displaystyle}
\setcounter{secnumdepth}{3}
%\voffset 3.8cm
\begin{document}
\DeclareGraphicsExtensions{.pdf,.png,.mps,.jpg}

\addcontentsline{toc}{chapter}{Apêndice 3}
\section*{Apêndice 3}

{\scriptsize
\begin{verbatim}
nsim = 1000; % numero de repetições
nint = 5000; % numero de jogadas
p = 100; % tamanho da população
tx = 1; % taxa de substituição

M = [3 0;5 1]; % matriz do jogo
Mr = [0.5 1;-1 0]; % matriz das reputações

R = zeros(p,1); % matriz de reputação.
Per = rand(p,1); % probabilidades de cooperar ou não cooperar de cada indivíduo.
P = zeros(p,1); % vetor de pontos por jogador
%SP = zeros(nint,1); % vetor de pontos por rodada
pes = rand(p,1); % pesos da media ponderada  entre as probabilidades de escolher e a probabilidade de cooperar dada por pela reputação.

c=[];
vint = (1:nint)';
vestab = zeros(nsim,1);
meanper = zeros(nsim,1);

co = [1,1]; % chute inicial para o ajuste exponencial
coe = [1,1]; % chute inicial para o ajuste exponencial
opts = optimset('Display','off');

for n = 1:nsim
    
    SP = zeros(nint,1); % vetor de pontos por rodada
    SR = zeros(nint,1);
    
    tr = randperm(p,tx*p); % substituição da população
    R(tr) = zeros(tx*p,1);
    Per(tr) = rand(tx*p,1);
    P(tr) = zeros(tx*p,1); % vetor de pontos por jogador
    
    pesa = rand(tx*p,1);
    
    if (n>1&&c(2)~=0)
        pesa = (1/c(2))*log(c(2).*pesa/c(1)+1); % Distribuição da nova geração
    end
    
    pes(tr) = pesa;
    
    for i = 1:nint
        
        js = randperm(p,2); % escolha dos jogadores 1 e 2
        
        ea = rand(1,1);
        eb = rand(1,1);
        
        ta = 1/(1+exp(-0.5*R(js(2)))); %% probabilidade de 1 cooperar pela reputação de 2
        tb = 1/(1+exp(-0.5*R(js(1))));%% probabilidade de 2 cooperar pela reputação de 1
        
        ma = (1-pes(js(1)))*Per(js(1))+(pes(js(1)))*ta; % media ponderada para a probabilidade de 1 cooperar
        mb = (1-pes(js(2)))*Per(js(2))+(pes(js(2)))*tb; % media ponderada para a probabilidade de 2 cooperar
        
        ta = ma; % troca de variavel
        tb = mb; % troca de variavel
        
        if ((ea<ta)&&(eb<tb))
            
            P(js) = P(js)+M(1,1);
            R(js) = R(js)+Mr(1,1);
            
            SR(i+1) = SR(i)+2*Mr(1,1);
            SP(i+1) = SP(i)+2*M(1,1);
            
        elseif((ea>=ta)&&(eb>=tb))
            
            P(js) = P(js)+M(2,2);
            R(js) = R(js)+Mr(2,2);
            
            SR(i+1) = SR(i)+2*Mr(2,2);
            SP(i+1) = SP(i)+2*M(2,2);
            
        elseif((ea<ta)&&(eb>=tb))
            
            P(js(1)) = P(js(1))+M(1,2);
            R(js(1)) = R(js(1))+Mr(1,2);
            P(js(2)) = P(js(2))+M(2,1);
            R(js(2)) = R(js(2))+Mr(2,1);
            
            SR(i+1) = SR(i)+Mr(1,2)+Mr(2,1);
            SP(i+1) = SP(i)+M(1,2)+M(2,1);
            
        elseif((ea>=ta)&&(eb<tb))
            
            P(js(1)) = P(js(1))+M(2,1);
            R(js(1)) = R(js(1))+Mr(2,1);
            P(js(2)) = P(js(2))+M(1,2);
            R(js(2)) = R(js(2))+Mr(1,2);
            
            SR(i+1) = SR(i)+Mr(1,2)+Mr(2,1);
            SP(i+1) = SP(i)+M(1,2)+M(2,1);
            
        end
        
    end
    
    SR = SR(2:end); SP = SP(2:end);
    
    A = [ones(size(R)) R]; b = A'*P; A = A'*A; coe = A\b;
    
    fun = @(x)(P-x(1).*exp(x(2).*pes)); % ajuste exponencial
    coe = lsqnonlin(fun,coe,[],[],opts); % ajuste exponencial
    
    [nind,xind] = hist(pes,10); % ajuste exponencial
    yind = nind.*(coe(1)*exp(coe(2)*xind)); % para o ajuste exponencial
    fun = @(x)(yind-x(1).*exp(x(2).*xind)); % ajuste exponencial
    co = lsqnonlin(fun,co,[],[],opts); % ajuste exponencial
    alf = co(2)/(co(1)*(exp(co(2))-1)); % ajuste exponencial
    c = [co(1)*alf co(2)]; % ajuste exponencial
    vestab(n) = SP(end)/(nint+1); % ajuste exponencial
    
    meanper(n,1) = mean(pes);
    
end

disp(['Retorno medio por jogada => ' num2str(SP(end)/(nint+1))])
disp(['Reputacao media por jogada => ' num2str(SR(end)/(nint+1))])
disp(' ')

f1 = figure(1);
subplot(2,3,1);
plot(vint,SR./vint)
title('Reputação média por jogada')

subplot(2,3,2);
plot(R,P,'*')
title('Reputação vs Pontos')
hold on
plot([min(R) max(R)],coe(1)+coe(2)*[min(R) max(R)])
hold off

subplot(2,3,3);
plot(vint,SP./vint)
title('Ganho Médio')

subplot(2,3,4);
plot(pes,P,'*')
title('Perfil vs Pontos')

subplot(2,3,5);
plot(xind,yind,'*',linspace(0,max(xind),100),co(1)*exp(co(2).*linspace(0,max(xind),100)))
hold on
plot(linspace(0,max(pes),100),coe(1)*exp(coe(2).*linspace(0,max(pes),100)))
hold off

f2 = figure(2);
subplot(2,1,1);
plot(1:nsim,vestab,'--*')
title('Ganho Médio por Simulação')

subplot(2,1,2);
plot(1:nsim,meanper)
title('Perfil Médio por Simulação')

\end{verbatim}}



%\end{document}